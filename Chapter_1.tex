\renewcommand{\bibname} %{Literature}
\setcounter{chapter}{0}

\chapter{Introduction}
\section{Thesis outline}

Motivation of the present research is to investigate influence of structure and annealing treatment on magnetic and transport properties of magnetic tunnel junctions (MTJs) with the special focus on spin transfer Torque (STT) and related phenomena. The present thesis mostly focuses on static and quasi-static spin-induced phenomena in MTJs with in-plane magnetization.
In this work I show that it is possible to design and develop multilayered MTJ nanostructures that allow to achieve switching of magnetic configuration, corresponding to low and high resistance states (TMR>100\%), using spin transfer torque effect exerted by spin-polarized current.

In the introduction and literature review part of this thesis general overview of this interesting and very promising phenomena is given. My original work will be presented chronologically. In the first part I try to show results from our study on optimization of MTJs of pseudo spin valve (PSV), and exchange bias spin valve (EB-SV) type, fabricated in INESC-MN (Lisbon, Portugal). The samples were deposited by PhD Piotr Wisniowski during his stay at INESC-MN and sent to Krakow for characterization. My task was to study magnetic properties of the samples %(N2XRK1-18)
at as-deposited and annealed stage. This part of work was done during 2007-08 academic year ($1^{st}$ year of PhD studies) and reported at Physics of Magnetism-08 international conference as a poster entitled \'Influence of annealing on crystallization, magnetic properties and magnetoresistance of spin valve MgO-based tunnel magnetoresistance junctions.
Obtained data on structural properties of this series of samples was done by PhD Jaroslaw Kanak and reported at several international conferences (Intermag-08, ISAMMA-11). 
Main results are the following.
1. Influence of annealing temperature $T_a$ was established for complete and incomplete PSV and EB-SV.
2. Macrospin calculations of M-H loops were performed. Magnetic properties of materials were obtained and analyzed for different $T_a$.
3. Correlation between x-ray diffraction (XRD) and vibrating sample magnetometry (VSM) -- structural and magnetic properties of thin films with transport properties of nanopillars was studied.

The next project focused on MTJs with Synthetic Antiferromagnet (SyAF or SAF) as a free layer (FL) was done in years 2008-2015. Our aim was to study the effect of varied Ruthenium (Ru) thickness and corresponding to it Interlayer Exchange Coupling (IEC) $J_{Ru}$ in FL-SAF on the critical current density $J_c$ in for STT-induced magnetization switching.
This project was done as a part of the European Research training Network SPINSWITCH in collaboration with INESC-MN. At February 2008 during my visit to this institute SAF structure was optimized and nanofabrication of MTJ-SAF was initiated. 
First generation of SAF trilayers were made by two different thin film deposition techniques: magnetron sputtering and Ion Beam Assisted Deposition (IBAD). From M-H VSM we observed that IBAD samples are better suited for MTJ-SAF study. Thus the decision to focus on MTJ-SAFs prepared by IBAD was taken. Two MTJ-SAF samples and one conventional MTJ sample (as a reference one) were nanofabricated. 

Results of the studies were presented in Krakow at Spinwork workshop (2008) as well as at two international conferences: SPINTECH V and ICM conferences in 2009 and appeared as a paper in JPCS'10.
It was proved that FL-SAF decreases $J_c$, in comparison with conventional FL. That was to our knowledge first CIMS using STT in MTJ-SAF in Poland.

However obtained TMR wasn't high enough for practical applications and we pursued further.

Next stage of the project aimed to obtain giant TMR and further optimize MTJ-SAF structure for STT-MRAM applications.
State-of-the-art thin films with MTJ-SAF multilayer were deposited by sputtering in the Singulus Timaris machine. Thin films stack had a continuous change of Ru thickness in the FL of MTJ (Ru wedge 0.50-1.05 nm) (First series had only several values of tRu that were controlled by sputtering time. No additional structural characterization on the INESC-MN MTJs was done in order to see correspondence between nominal and actual $t_{Ru}$).
Samples were initially characterized by magnetooptic Kerr effect  MOKE and VSM. Then three Ru thicknesses, corresponding to strong, medium, and low coupling in FL-SAF were chosen for nanofabrication. It was performed by PhD Witold Skowronski at the University of Bielefeld (Germany) using top-down approach, e-beam lithography and subsequent ion etching. MTJs nanopillars had common bottom electrode and individual gold-covered top one. Minimal pillar planar sizes were 115 nm x 230 nm.
Transport measurement revealed that obtained nanopillars have way too high coercivity that makes them hard to switch using STT effect alone. Thus, additional magnetic field needs to be applied in order to lower energy barrier that is necessary for switching using STT effect. High coercivity, however is an advantage for memory applications, because it guarantees longer "memory time" due to high thermal stability factor $\Delta=\frac{E_0}{k_{BT}}$ (One of non-volatility requirements is that thermal stability $\Delta$ should be higher than 50).
Several ways to determine $\Delta$ in MTJ-SAF were performed. First $J_c$ as a function of pulse time or applied field was measured that allowed to derive $\Delta$. Both ways resulted in high enough values of $\Delta$ that together with high TMR and low RA product properties make MTJ-SAF suitable for applications.
Results were presented as an oral talk at JEMS'10 conference and as a poster at Nanosmat'11.

During PhD studies several projects that aimed for deeper understanding of MTJ optimization and STT-related phenomena were done.
The most important one dealt with suggesting a model for detection of pinhole contribution to conductance of MgO-wedge MTJs samples with varied barrier thickness. Suggested approach was based on extension of  equivalent circuit model by introducing weight coefficients to the conductance channels. Results are published in Journal of Applied Physics in the article \textit{The study of conductance in magnetic tunnel junctions with a thin MgO barrier: the effect of Ar pressure on TMR and RA product}.

Another project aimed at the explanation of the FM IEC in sputtered CoFeB/MgO/CoFeB MTJs. In the framework of free-electron Slonczewski model \cite{Slo89} no suitable explanation was found, signalizing that this model can not be used for MTJs with very thin MgO barrier (below 1 nm). 6-dimensional parameter space was analyzed and some projections were visualized in order to find appropriate theoretical parameters that may explain obtained experimental data.
Results were presented as a poster at JEMS 2010.

Two more satellite projects were done that are outside the scope of the present thesis but are worth mentioning.
1)	Study of domain wall velocity($V_{DW}$) in microstructured samples of CoFeB on different buffers. 
Samples had 40 micrometer wide wire and injection pad for domain wall generation.
TR- MOKE and micro-MOKE were done. Data processing allowed to obtain that no significant change in $V_{DW}$ for different buffer structures is observed. 
2)	Search for current-indiced out-of-plane precession of magnetization (in zero-field) in the FL of CPP-GMR Dual Spin Valves (DSV) with Wavy Angular Dependence of STT.
Two generations of DSV were nanofabricated. No precession was found in either of them. Possible reasons: low GMR signal, not optimized nanofabrication process; deviation from macrospin behaviour at high currents (decoherence) \cite{Par91,Slo89}.



\begin{thebibliography}{Aok99}

\bibitem[Par 91]{Par91} S. S. P. Parkin et al. Phys. Rev. Lett. 67, 3598-3601 (1991)
\bibitem[Slo 89]{Slo89} J.C. Slonczewski Phys. Rev. B 39, 6995 (1989)
\end{thebibliography}